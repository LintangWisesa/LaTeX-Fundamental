\documentclass{article}
\usepackage{circuitikz}

\begin{document}
    
    % PATH METHOD
    % \draw (x1,y1) to[namakomponen] (x2,y2);
    \begin{circuitikz}
        \draw (0,0) to[resistor] (3,0);
        \draw (0,0) to[resistor] (0,-3);
    \end{circuitikz}

    \vspace{0.3in}

    % NODE METHOD
    % \draw (x,y) node[namakomponen](var){label};
    \begin{circuitikz}
        \draw (1.5,0) node[resistorshape](R){}
        (R.left) --(0,0)
        (R.right) --(3,0);
    \end{circuitikz}

    \section{Symbol}

    \subsection{Resistor}

    \begin{enumerate}

        % ########################################
        % BASIC RESISTOR
        \item Basic Resistor
        
        \begin{circuitikz}[american]
            \draw (0,0) to[resistor] (2,0);
            \draw (2.5,0) to[R] (4.5,0);
            \draw (5,0) to[R, l=$R_1$] (7,0);
            \draw (7.5,0) to[R, a^=$5\Omega$] (9.5,0);
        \end{circuitikz}

        \vspace{0.3in}
        
        \begin{circuitikz}[european]
            \draw (0,0) to[resistor] (2,0);
            \draw (2.5,0) to[R] (4.5,0);
            \draw (5,0) to[R, a=$100\Omega$] (7,0);
            \draw (7.5,0) to[R=$10\Omega$] (9.5,0);
        \end{circuitikz}

        \vspace{0.3in}

        \begin{circuitikz}
            \draw (0,0) to[R=$133\Omega$, i=$10mA$, v=$3mV$] (3,0);
        \end{circuitikz}
        \begin{circuitikz}[european]
            \draw (3.5,0) to[R=$133\Omega$, i=$10mA$, v=$3mV$] (6.5,0);
        \end{circuitikz}

        % ########################################
        % VARIABLE RESISTOR
        \vspace{0.3in}
        \item Variable Resistor
        
        \begin{circuitikz}
            \draw (0,0) to[variable resistor] (2,0);
            \draw (2.5,0) to[vR, l=$vR$] (4.5,0);
        \end{circuitikz}

        \vspace{0.3in}

        \begin{circuitikz}[european]
            \draw (0,0) to[variable resistor] (2,0);
            \draw (2.5,0) to[vR, a=$15\Omega$] (4.5,0);
        \end{circuitikz}

        % ########################################
        % POTENTIOMETER
        \vspace{0.3in}
        \item Potentiometer
        
        \begin{circuitikz}
            \draw (0,0) to[potentiometer] (2,0);
            \draw (2.5,0) to[pR, l=$vR$] (4.5,0);
        \end{circuitikz}

        \vspace{0.3in}

        \begin{circuitikz}[european]
            \draw (0,0) to[potentiometer] (2,0);
            \draw (2.5,0) to[pR, a=$15\Omega$] (4.5,0);
        \end{circuitikz}

        % ########################################
        % PHOTORESISTOR
        \vspace{0.3in}
        \item Photoresistor
        
        \begin{circuitikz}
            \draw (0,0) to[photoresistor] (2,0);
            \draw (2.5,0) to[phR, l=$vR$] (4.5,0);
        \end{circuitikz}

        \vspace{0.3in}

        \begin{circuitikz}[european]
            \draw (0,0) to[photoresistor] (2,0);
            \draw (2.5,0) to[phR, a=$15\Omega$] (4.5,0);
        \end{circuitikz}

        % ########################################
        % THERMISTOR
        \vspace{0.3in}
        \item Thermistor
        
        \begin{circuitikz}
            \draw (0,0) to[thermistor] (2,0);
            \draw (2.5,0) to[thR, l=$vR$] (4.5,0);
            \draw (5,0) to[thermistor ptc] (7,0);
            \draw (7.5,0) to[thermistor ntc] (9.5,0);
        \end{circuitikz}

        \vspace{0.3in}

        \begin{circuitikz}[european]
            \draw (0,0) to[thermistor] (2,0);
            \draw (2.5,0) to[thR, a=$15\Omega$] (4.5,0);
            \draw (5,0) to[thRp] (7,0);
            \draw (7.5,0) to[thRn] (9.5,0);
        \end{circuitikz}

        % ########################################
        % VARISTOR
        \vspace{0.3in}
        \item Varistor
        
        \begin{circuitikz}
            \draw (0,0) to[varistor] (2,0);
            \draw (0,0) to[varistor, a=$10\Omega$] (2,0);
        \end{circuitikz}

    \end{enumerate}

    \newpage

    \subsection{Kapasitor}

        \begin{enumerate}

            % ########################################
            % BASIC CAPACITOR
            \item Basic Capacitor
            
            \begin{circuitikz}
                \draw (0,0) to[capacitor] (2,0);
                \draw (2.5,0) to[C, l=$C_1$] (4.5,0);
                \draw (5,0) to[C, a=$100 \mu F$] (7.5,0);
                \draw (8,0) to[C, a^=$100 \mu F$] (10,0);
            \end{circuitikz}

            % ########################################
            % POLAR CAPACITOR
            \item Polar Capacitor
            
            \begin{circuitikz}
                \draw (0,0) to[polar capacitor] (2,0);
                \draw (2.5,0) to[pC, l=$C_1$] (4.5,0);
                \draw (5,0) to[pC, a=$100 \mu F$] (7.5,0);
                \draw (8,0) to[pC, a^=$100 \mu F$] (10,0);
            \end{circuitikz}

            % ########################################
            % ELECTROLYTHIC CAPACITOR
            \item Electrolythic Capacitor
            
            \begin{circuitikz}
                \draw (0,0) to[ecapacitor] (2,0);
                \draw (2.5,0) to[eC, l=$C_1$] (4.5,0);
                \draw (5,0) to[eC, a=$100 \mu F$] (7.5,0);
                \draw (8,0) to[eC, a^=$100 \mu F$] (10,0);
            \end{circuitikz}

            % ########################################
            % VARIABLE CAPACITOR
            \item Variable Capacitor
            
            \begin{circuitikz}
                \draw (0,0) to[variable capacitor] (2,0);
                \draw (2.5,0) to[vC, l=$C_1$] (4.5,0);
                \draw (5,0) to[vC, a=$100 \mu F$] (7.5,0);
                \draw (8,0) to[vC, a^=$100 \mu F$] (10,0);
            \end{circuitikz}

        \end{enumerate}

    \newpage
    \subsection{Inductor}

        \begin{enumerate}
            % ############################
            % BASIC INDUCTOR
            \item Basic Inductor
        
            \begin{circuitikz}
                \draw (0,0) to[inductor] (2,0);
                \draw (2.5,0) to[L, l=$L_1$] (4.5,0);
                \draw (5,0) to[L, a=$1 H$] (7,0);
            \end{circuitikz}

            \vspace{0.1in}

            \begin{circuitikz}[american]
                \draw (0,0) to[inductor] (2,0);
                \draw (2.5,0) to[L, l=$L_1$] (4.5,0);
                \draw (5,0) to[L, a=$1 H$] (7,0);
            \end{circuitikz}

            \vspace{0.1in}

            \begin{circuitikz}[european]
                \draw (0,0) to[inductor] (2,0);
                \draw (2.5,0) to[L, l=$L_1$] (4.5,0);
                \draw (5,0) to[L, a=$1 H$] (7,0);
            \end{circuitikz}

            % ############################
            % VARIABLE INDUCTOR
            \item Variable Inductor

            \begin{circuitikz}
                \draw (0,0) to[variable inductor] (2,0);
                \draw (2.5,0) to[vL, l=$L_1$] (4.5,0);
                \draw (5,0) to[vL, a=$1 H$] (7,0);
            \end{circuitikz}

            \vspace{0.1in}

            \begin{circuitikz}[american]
                \draw (0,0) to[variable inductor] (2,0);
                \draw (2.5,0) to[vL, l=$L_1$] (4.5,0);
                \draw (5,0) to[vL, a=$1 H$] (7,0);
            \end{circuitikz}

            \vspace{0.1in}

            \begin{circuitikz}[european]
                \draw (0,0) to[variable inductor] (2,0);
                \draw (2.5,0) to[vL, l=$L_1$] (4.5,0);
                \draw (5,0) to[vL, a=$1 H$] (7,0);
            \end{circuitikz}

        \end{enumerate}
    
    \newpage
    \subsection{Dioda}
        \begin{enumerate}
            
            % ##########################
            % BASIC DIODA
            \item Basic Dioda
            
            \begin{circuitikz}
                \draw (0,0) to[diode] (2,0);
                \draw (2.5,0) to[Do, l=$Do$] (4.5,0);
            \end{circuitikz}
            
            \vspace{0.1in}

            \begin{circuitikz}
                \draw (0,0) node[left]{Anode} 
                to[Do, l=$diode$] (2,0)
                node[right]{Cathode};
            \end{circuitikz}

            \vspace{0.1in}

            \begin{circuitikz}
                \draw (0,0) to[full diode] (2,0);
                \draw (2.5,0) to[D*, l=$Do$] (4.5,0);
            \end{circuitikz}

            % #############################
            % LED Light Emitting Diode
            \item LED
            
            \begin{circuitikz}
                \draw (0,0) to[led] (2,0);
                \draw (2.5,0) to[leDo] (4.5,0);
                \draw (5,0) to[full led] (7,0);
                \draw (7.5,0) to[leD*] (9.5,0);
            \end{circuitikz}

            % #############################
            % PhotoDiode
            \item Photodiode
            
            \begin{circuitikz}
                \draw (0,0) to[photodiode] (2,0);
                \draw (2.5,0) to[pDo] (4.5,0);
                \draw (5,0) to[full photodiode] (7,0);
                \draw (7.5,0) to[pD*] (9.5,0);
            \end{circuitikz}

            % #############################
            % Varicap Diode
            \item Varicap diode
            
            \begin{circuitikz}
                \draw (0,0) to[varcap] (2,0);
                \draw (2.5,0) to[VCo] (4.5,0);
                \draw (5,0) to[full varcap] (7,0);
                \draw (7.5,0) to[VC*] (9.5,0);
            \end{circuitikz}

            % #############################
            % Tunnel Diode
            \item Tunnel diode
            
            \begin{circuitikz}
                \draw (0,0) to[tunnel diode] (2,0);
                \draw (2.5,0) to[tDo] (4.5,0);
                \draw (5,0) to[full tunnel diode] (7,0);
                \draw (7.5,0) to[tD*] (9.5,0);
            \end{circuitikz}

            % #############################
            % Zenner Diode
            \item Zenner diode
            
            \begin{circuitikz}
                \draw (0,0) to[ZZener diode] (2,0);
                \draw (2.5,0) to[zzDo] (4.5,0);
                \draw (5,0) to[full ZZener diode] (7,0);
                \draw (7.5,0) to[zzD*] (9.5,0);
            \end{circuitikz}

            % #############################
            % Schottky Diode
            \item Schottky diode
            
            \begin{circuitikz}
                \draw (0,0) to[Schottky diode] (2,0);
                \draw (2.5,0) to[sDo] (4.5,0);
                \draw (5,0) to[full Schottky diode] (7,0);
                \draw (7.5,0) to[sD*] (9.5,0);
            \end{circuitikz}
        
        \end{enumerate}

    \newpage
    \subsection{Transistor}

    \begin{enumerate}
        
        % ########################
        % NPN TRANSISTOR
        \item Transistor NPN
        
        \begin{circuitikz}
            \draw (0,0) node[npn](){};

            \draw (3,0) node[npn](T1){$T_1$};
            \draw (T1.E) node[below]{Emitter};
            \draw (T1.C) node[above]{Collector};
            \draw (T1.B) node[left]{Base};

            \draw (7,0) node[npn](T2){$T_2$}
            (T2.B) to[R] (5,0);
        \end{circuitikz}

        % ########################
        % PNP TRANSISTOR
        \item Transistor PNP
        
        \begin{circuitikz}
            \draw (0,0) node[pnp](){};

            \draw (3,0) node[pnp](T1){$T_1$};
            \draw (T1.E) node[above]{Emitter};
            \draw (T1.C) node[below]{Collector};
            \draw (T1.B) node[left]{Base};

            \draw (7,0) node[pnp](T2){$T_2$}
            (T2.B) to[R] (5,0);
        \end{circuitikz}

        % ########################
        % NPN PHOTOTRANSISTOR
        \item PhotoTransistor NPN
        
        \begin{circuitikz}
            \draw (0,0) node[npn, photo](){};

            \draw (3,0) node[npn, photo](T1){$T_1$};
            \draw (T1.E) node[below]{Emitter};
            \draw (T1.C) node[above]{Collector};
            \draw (T1.B) node[left]{Base};
        \end{circuitikz}

        % ########################
        % PNP PHOTOTRANSISTOR
        \item PhotoTransistor PNP
        
        \begin{circuitikz}
            \draw (0,0) node[pnp, photo](){};

            \draw (3,0) node[pnp, photo](T1){$T_1$};
            \draw (T1.E) node[above]{Emitter};
            \draw (T1.C) node[below]{Collector};
            \draw (T1.B) node[left]{Base};
        \end{circuitikz}

        % ########################
        % MOSFET: metal oxide semiconductor FET field effect transistor
        \item MOSFET
        
        \begin{circuitikz}
            \draw (0,0) node[nmos](nMOS){$nMOS$};
            \draw (3,0) node[pmos](pMOS){$pMOS$};
        \end{circuitikz}
    
    \end{enumerate}

    \newpage
    \subsection{Transformer}

    \begin{enumerate}
        
        \item Transformer
        
        \begin{circuitikz}
            \draw (0,0) node[transformer](T1){};
        \end{circuitikz}

        \begin{circuitikz}[american]
            \draw (0,0) node[transformer](T2){};
        \end{circuitikz}
        
        \begin{circuitikz}[european]
            \draw (0,0) node[transformer](T3){};
        \end{circuitikz}

        \begin{circuitikz}
            \draw (0,0) node[transformer](T4){};
            \draw (T4.base) node{$T_1$};
            \draw (T4.A1) node[left]{$A_1 (P_1)$};
            \draw (T4.A2) node[left]{$A_2 (P_2)$};
            \draw (T4.B1) node[right]{$B_1 (S_1)$};
            \draw (T4.B2) node[right]{$B_2 (S_2)$};
        \end{circuitikz}

    \end{enumerate}

    \newpage
    \subsection{Measurement tools}

    \begin{enumerate}

        \item Amperemeter
        
        \begin{circuitikz}
            \draw (0,0) to[ammeter] (2,0);        
        \end{circuitikz}

        \item Voltmeter
        
        \begin{circuitikz}
            \draw (0,0) to[voltmeter] (2,0);        
        \end{circuitikz}

        \item Ohmmeter
        
        \begin{circuitikz}
            \draw (0,0) to[ohmmeter] (2,0);        
        \end{circuitikz}
    
    \end{enumerate}

    \subsection{Sources}
    
    \begin{enumerate}
        
        \item Basic Source
        
        \begin{circuitikz}
            \draw (0,0) to[voltage source, l=$V$] (2,0);
            \draw (2.5,0) to[current source, l=$I$] (4.5,0);
        \end{circuitikz}

        \begin{circuitikz}[american]
            \draw (0,0) to[voltage source, l=$V$] (2,0);
            \draw (2.5,0) to[current source, l=$I$] (4.5,0);
        \end{circuitikz}

        \begin{circuitikz}[european]
            \draw (0,0) to[voltage source, l=$V$] (2,0);
            \draw (2.5,0) to[current source, l=$I$] (4.5,0);
        \end{circuitikz}

        \item Battery
        
        \begin{circuitikz}
            \draw (0,0) to[battery, l=$V$, a=$3V$] (2,0);
            \draw (2.5,0) to[battery1, l=$V$, a=$1.5V$] (4.5,0);
            \draw (5,0) to[battery2, l=$V$, a=$1.5V$] (7,0);
        \end{circuitikz}

        \item DC Source
        
        \begin{circuitikz}
            \draw (0,0) to[dcvsource, l=$V_\textrm{{\tiny{DC}}}$] (2,0);
            \draw (2.5,0) to[dcisource, l=$I_\textrm{{\tiny{DC}}}$] (4.5,0);
            \draw (5,0) to[sinusoidal voltage source, l=$V_\textrm{{\tiny{AC}}}$] (7,0);
        \end{circuitikz}

    \end{enumerate}

    \newpage
    \subsection{Switch}

    \begin{enumerate}
        
        \item Close SPST: single pull single throw
        
        \begin{circuitikz}
            \draw (0,0) to[switch, l=$\textrm{\tiny{switch}}$] (2,0);
            \draw (2.5,0) to[spst, l=$\textrm{\tiny{spst}}$] (4.5,0);
            \draw (5,0) to[closing switch, l=$\textrm{\tiny{closing switch}}$] (7,0);
            \draw (7.5,0) to[cspst, l=$\textrm{\tiny{cspst}}$] (9.5,0);
        \end{circuitikz}

        \item Open SPST: single pull single throw
        
        \begin{circuitikz}
            \draw (0,0) to[nos, l=$\textrm{\tiny{nos}}$] (2,0);
            \draw (2.5,0) to[normal open switch, l=$\textrm{\tiny{normal open switch}}$] (4.5,0);
            \draw (5,0) to[opening switch, l=$\textrm{\tiny{opening switch}}$] (7,0);
            \draw (7.5,0) to[ospst, l=$\textrm{\tiny{ospst}}$] (9.5,0);
        \end{circuitikz}

        \item Normal close switch
        
        \begin{circuitikz}
            \draw (0,0) to[normal closed switch] (2,0);
            \draw (2.5,0) to[ncs] (4.5,0);
        \end{circuitikz}

        \item Push button
        
        \begin{circuitikz}
            \draw (0,0) to[push button] (2,0);
        \end{circuitikz}

        \item Toggle switch
        
        \begin{circuitikz}
            \draw (0,0) to[toggle switch] (2,0);
        \end{circuitikz}

        \item SPDT: single pull double throw
        
        \begin{circuitikz}
            \draw (0,0) node[spdt](S){};
            \draw (2,0) node[spdt](S1){}
            (S1.in) node[left] {in}
            (S1.out 1) node[right] {out 1}
            (S1.out 2) node[right] {out 2};
        \end{circuitikz}

    \end{enumerate}

    \newpage
    \subsection{Gerbang Logika}

    \vspace{0.1in}

    \begin{circuitikz}[american]
        \draw (0,0) node[and port]{\tiny{and}};
        \draw (1.5,0) node[or port]{\tiny{or}};
        \draw (3,0) node[nand port]{\tiny{nand}};
    \end{circuitikz}

    \vspace{0.1in}

    \begin{circuitikz}[american]
        \draw (4.5,0) node[nor port]{\tiny{nor}};
        \draw (6,0) node[xor port]{\tiny{xor}};
        \draw (7.5,0) node[xnor port]{\tiny{xnor}};
    \end{circuitikz}

    \vspace{0.1in}

    \begin{circuitikz}[european]
        \draw (0,0) node[and port]{};
        \draw (2,0) node[or port]{};
        \draw (4,0) node[nand port]{};
    \end{circuitikz}

    \vspace{0.1in}
    
    \begin{circuitikz}[european]
        \draw (6,0) node[nor port]{};
        \draw (8,0) node[xor port]{};
        \draw (10,0) node[xnor port]{};
    \end{circuitikz}

    \newpage
    \subsection{Draw a circuit}

    Cara menggambar rangkaian elektronika di \LaTeX

    \vspace{0.2in}

    % resistor R1 seri R2, kemudian paralel R3
    % CARA 1
    \begin{circuitikz}
        \draw (0,0) to[R=$R_1$] (2,0);
        \draw (2,0) to[R=$R_2$] (4,0);
        \draw (0,0) to (0,-1);
        \draw (4,0) to (4,-1);
        \draw (0,-1) to[R, a=$R_3$] (4,-1);
    \end{circuitikz}

    % CARA 2
    \begin{circuitikz}
        \draw (0,0) to[R=$R_1$] (2,0)
        (2,0) to[R=$R_2$] (4,0)
        (0,0) to (0,-1)
        (4,0) to (4,-1)
        (0,-1) to[R, a=$R_3$] (4,-1);
    \end{circuitikz}

    % CARA 3
    \begin{circuitikz}
        \draw (0,0) to[R=$R_1$] (2,0)
        (2,0) to[R=$R_2$] (4,0) -- (4,-1)
        to[R, a=$R_3$] (0,-1) -- (0,0);
    \end{circuitikz}

    \vspace{0.2in}

    % baterai & LED
    \begin{circuitikz}
        \draw (0,0) to[battery, l=$V_1$] (0,2)
        (2,1) node[emptylediodeshape, rotate=90](L){}
        (L.right) |- (0,2)
        (L.left) |- (0,0);
    \end{circuitikz}

    \vspace{0.2in}

    % baterai & LED beserta amperemeter & voltmeter
    \begin{circuitikz}
        \draw (0,0) to[battery, l=$V_1$] (0,2)
        (0,2) to[ammeter] (2,2)
        (2,0) to[led] (2,2)
        (0,0) -- (2,0)
        (2,0.3) -- (3.3,0.3)
        (2,1.7) -- (3.3,1.7)
        (3.3,0.3) to[voltmeter] (3.3,1.7);
    \end{circuitikz}

\end{document}